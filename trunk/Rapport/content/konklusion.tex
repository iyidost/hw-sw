\chapter{Konklusion}\label{cha:konklusion}

Projektet var hårdt at komme i gang med, da det har været svært at begynde et fornuftigt sted. Hele lc3 systemet skulle opbygges fra bunden, dette gjorte dog samtidig at igennem en stor arbejdsindsat opnået en større forståelse for interaktion mellem hardware og software. Især opbygningen af en computer og hvordan de forskellige moduler i en computer fungerer og kommuniker sammen. Det der var mest problematisk ved opstarten af projektet, var forståelsen af hvilke moduler der var nødvendige for at kunne designet en simpel computer. Dette blev løst ved hjælp af samtale med underviser, hjælpelærer og forelæsninger. Derefter påbegyndes arbejdet med at designet af de forskellige moduler til minimumsimplementeringen. Minimumsimplementeringen var implementeret i slutningen af uge 2, dette gav tid til at videreudvikle projektet. De punkter der hovedsageligt efterfølgende blev forbedret var VGA og C koden. Dette har givet større indsigt i VGA low level.



Vi har med dette projekt fået en bedre forståelse for sammenspillet mellem hardware og software, hvordan de nødvendige komponenter i LC3'en bruges i samspil med et stykke software - her vores simulator spil. Vi har udviklet os fremadrettet mht. at forstå hvad der egentligt sker på komponent niveau når man bruger en computer, der kører noget software. vi føler os bedre til at kunne abstrahere på forskellige niveauer mht. at "dykke ned" på/i komponent niveau og få et sådant til at fungere, i sammenspil med et stykke software (her vores simulator spil) og bevæge os op og ned mellem de forskellige abstraktions lag i henhold til low-level/high-level udvikling. Det har været meget interessant at lave/kode de respektive komponenter(så som memory, VGA.. etc) vi har brugt og derefter få dem til at "snakke" sammen med vores spil.
Det har været et meget interessant projekt at gå i krig med, det har givet en meget bedre forståelse for hvad der egentligt sker "inden i" en computer, når man som bruger "bare" sidder foran skærmen og bruger den til div. ting. Vores viden er blevet bredere efter dette projekt.
Dog er tidsfaktoren en afgørende faktor for udviklingen af produktet kontra ønskede opnåede mål, vi forestiller os at man ude i erhvervslivet har mere tid til udviklingen, men vi er generelt tilfredse med det vi har nået på den givne tid.. :) 