\section{Ram}
 LC3 complete system RAM implementering.
Vi har valgte at bruge single-port synchronous RAM. Vi valgte single-port RAM fordi vi i første omgang ikke har brug for at skrive til eller læse fra 2 adresser samtidig, som duel-port RAM ville have givet os lov til. Synchronous tillader at vi bruge block rammen på FPGA boardet, og ikke CLBs rammen, der er forbeholdt til logic og asynchronous RAM. Til det vi skulle lave var der ingen grund til at bruge asynchronous, da alle vores udregninger ikke behøver at blive behandlet sammen clock cycle. Vi implementerede vores RAM, ved at følge bogen (indsæt eksempel/reference) og derefter justere længden af adressen så det passede til den størrelse vores program der blev kørt på FPGA'en havde, ved at reducere størrelsen på rammen, sørgede vi for at syntetiseringen af vores LC3 system ikke tog unødvendig langt tid. Vi blev dog løbende nød til at ændre på adressen længden af hukommelsen, da vores C kode blev længere. Hvad bruger vi rammen til ud over vores c kode?  

 Vi valgte til vores Ram at lave en Wrapper, der indeholder en tristatebuffer og oversætter signalet til den rigtige længde.
Tristate Bufferen sørger for at signal kun kan gå den ene vej, så der enten kan blive læste fra Rammen eller skrevet til den på en given clock cycle. Dette indeholder alle vores elementer, og dette gør sammen med kontrol på bus adressen at vi kun kan skrive eller læse fra det vi ønsker.
Padding er den måde vi løser at vi har en bus adresse på 16 bit, men i vores hukommelse kun bruger eksempelvis 14 bit, så vælger vi at bruge de 14 mindst betydende bit når hukommelsen modtager en bus adresse. Hvis hukommelsen sender en adresse til bussen, er denne for kort, dette løses ved at sætte 0'er foran, så vi ender med '00' og adressen fra hukommelsen.