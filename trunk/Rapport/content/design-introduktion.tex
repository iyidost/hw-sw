Det samleste system som overordnet er beskrevet på figur~\vref{fig:Diagrammer/OverordnetSystem}, er opbygget af et \texttt{FPGA board} der implementerer en computer med en black box \texttt{LC3 CPU}. Til computeren er der af eksterne I/O komponenter tilknyttet en skærm og et I/O board\footnote{Digilent DIO4\texttrademark I/O board}, til et udvidelses board til FPGA'en er der desuden tilsluttet et rat som essentielt er en variabel modstand. Desuden benyttes der to-vejs seriel kommunikation til en PC til upload af bruger-programmer til \texttt{LC3} computeren samt verificering af disse, og derudover afsending af scores til highscore applikationen på computeren.

\billede{!htbp}{0.5}{Diagrammer/OverordnetSystem}{Design system}

Spillogikken er skrevet på PC i C kode og derefter overført til LC3 computerens RAM på FPGA boardet ved hjælp af en serial forbindelse. Indholdet af spillet gemmes i block ram på FPGA. En VGA skærm viser output og er opdateret løbende, for at man kan se og følge med i hvad der sker og dermed kan styre spillet.

LC-3 system design består af forskellige komponenter som er sat sammen. De forskellige komponenter er CPU, rat I/O, Memory (RAM og ROM), UART, VGA og DIO-4 board.

Til dette projekt har det været nødvendigt at tilkoble en VGA skærm og et rat via en A/D converter.