Vores system som overordnet er beskrevet på figur~\vref{fig:Diagrammer/OverordnetSystem}, er opbygget af et \texttt{FPGA board} der implementerer en computer kørt af en \texttt{LC3 CPU}. Til computeren er der af eksterne I/O komponenter tilknyttet en skærm, et I/O board\footnote{Digilent DIO4\texttrademark I/O board} samt et rat som essentielt er en variabel modstand. Desuden benyttes der to-vejs seriel kommunikation til en PC til upload af bruger programmer til \texttt{LC3} computeren samt verificering af disse.

\billede{!htbp}{1}{Diagrammer/OverordnetSystem}{Design system}

Spillet er først implementeret på PC i C kode, derefter skal indholdet overføres til FPGA board vha. en serial forbindelse. Indholdet af spillet gemmes i block ram på FPGA. En VGA skærm viser "output" altså baggrund,forgrund( bil, forhindringer), og er opdateret løbende, for at man kan se og følge med i hvad det sker og dermed kan spille. Spillet kan styres direkte fra FPGA board vha. en styrrings enhed, i dette tilfælde har vi valgt at bruge et rat i stedet for et keyboard.

LC-3 system efter vores design består af forskellige komponenter som er sat sammen, dette kan tælles: CPU, rat, MEM, UART og forbindende ”busser”.

Udover VGA skærm kan man koble flere hardware enheder til når der er behov for det. De kan være lydkort,... o.s.v.
CPU har til formål at udføre de beregninger som ligger til grund for hvilke instruktioner som skal udføres, dvs. hvad applikationen ”siger” der skal ske hvornår....
MEM bruges generelt til at gemme div. Data, bl.a. de respektive instruktioner som skal udføres ...
UART bruges til at ”holde” styr på at sende/modtage data instruktioner til/fra komponenterne i computeren ....
rattet bruges til at styre bilen, som kører i forgrunden.
Figur xx viser vores overordnede system.
(OBS! Vi skal selv tegne LC-3 system diagram som ligner denne figur)