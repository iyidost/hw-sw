\chapter{Design}\label{cha:design}
Vores system som overordnet er beskrevet på figur~\vref{fig:Diagrammer/OverordnetSystem}, er opbygget af et \texttt{FPGA board} der implementerer en computer kørt af en \texttt{LC3 CPU}. Til computeren er der af eksterne I/O komponenter tilknyttet en skærm, et I/O board\footnote{Digilent DIO4\texttrademark I/O board} samt et rat som essentielt er en variabel modstand. Desuden benyttes der to-vejs seriel kommunikation til en PC til upload af bruger programmer til \texttt{LC3} computeren samt verificering af disse.

\billede{!htbp}{1}{Diagrammer/OverordnetSystem}{Design system}

Spillet er først implementeret på PC i C kode, derefter skal indholdet overføres til FPGA board vha. en serial forbindelse. Indholdet af spillet er gemmes i block ram på FPGA. En VGA skærm viser "output" altså baggrund,forgrund( bil, forhindringer), og er opdateret løbende, for man kan se og følge med hvad det sker og dermed kan spille. Spillet kan styres direkt fra FPGA board vha. en styrring enhed, i denne tilfælde har vi valgt at bruge en rat i stedet for en keyboard.

LC-3 system efter vores design består af forskellige komponenter som er sat sammen, dette kan tælles: CPU, rat, MEM, UART og  forbindende busser . Udover VGA skærm kan man kobler flere hardware enheder til når der er behøv for. De kan være lydkort,... o.s.v.

CPU har til formålet bruges til ....
MEM bruges til ...
UART bruges til ....
rat bruges til at styre billen som køre på forgrund.

Figure xx vises vores overordnet system.
(OBS! Vi skal selv tegne LC-3 system diagram som ligner denne figur)