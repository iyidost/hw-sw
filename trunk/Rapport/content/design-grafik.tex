\section{Grafik}
Til projektet efter minimum implementeringen var færdiggjort, blev grafikken til spillet udvidet. 

% Tilføj værktøjer til appendix og referere til dem herfra
% Tilføj værktøjer til appendix og referere til dem herfra
% Tilføj værktøjer til appendix og referere til dem herfra
% Tilføj værktøjer til appendix og referere til dem herfra
% Tilføj værktøjer til appendix og referere til dem herfra
% Tilføj værktøjer til appendix og referere til dem herfra
% Tilføj værktøjer til appendix og referere til dem herfra

\subsection{Tiles}
Tiles blev lavet ved hjælp af værktøjer\footnote{http://matisen.dk/dtu/02321/tile\_maker.php}, udarbejdet i forbindelse med projektet. De genererede hex arrays blev lagt ind i en Tile ROM. Værdierne 0 eller 1 arrayet bruges til at bestemme farven på hver pixel. I anden ROM er forgrunds og baggrundsfarve defineret for hver tile, disse farver er af 8 bits længde da der zero paddes.

\subsection{Sprites}
Sprites blev tegnet i Adobe Photoshop og gemt i png format, derefter blev de konverteret til et hex array ved hjælp af online værktøjer\footnote{http://matisen.dk/dtu/02321}, udviklet i forbindelse med projektet. Disse Hex arrays blev lagt ind i en VGA\_ROM som tegnede rammen for objektet, derefter blev der ud fra png filen lave en inverted udgave som blev gemt i en anden ROM som udgjorte masken til objektet. Til hver af disse ROM der tildelt 1 farve hvor de værdier der er sat til 1 fik denne farve. Ved hjælp af denne metode kunne rammen og masken få 2 forskellige farver der danne et objekt med 2 forskellige 24 bit farver.