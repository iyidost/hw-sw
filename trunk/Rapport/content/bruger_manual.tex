\section{Bruger manual}
\subsection*{Før du kan afprøve spillet}
\subsubsection*{Upload af firmware til FPGA Boardet}
Det forventes at XILINX\textregistered miljøet er installeret på computeren på forhånd og FPGA boardet er tilsluttet til computeren via USB og Seriel forbindelse.
Firmware uploades dernæst til FPGA boardet ved at benytte den genererede .bit fil. Højreklik på filen og vælg "Upload to FPGA board".

\subsubsection*{Start highscore applikationen}
Start nu \textbf{PC\_Highscore.exe}, denne applikation vil nu kommunikere med FPGA boardet og registrere scores.

\subsection*{Ekstra: Opdater software på LC3 computeren}
Skulle du få brug for at installere opdateret software på LC3 computeren gøres dette ved at højreklikke på \texttt{.obj} filen og vælge "Run on FPGA", derefter genstarter LC3 computeren og den opdaterede software benyttes nu.

Bemærk at software opdateringen igen skal installeres hvis FPGA boardet har været slukket.

\billede{!htbp}{0.5}{photo}{Billede af kørende spil}
\billede{!htbp}{0.5}{photo-3}{Billede af tilsluttet rat}

\subsection*{Sådan spiller du \textbf{Speed Devil}}
Bilen styres ved at dreje på rattet. Formålet med spillet er at undgå sammenstød med elementer på vejen, i tilfælde af man rammer en af forhindringerne er spillet slut og Game Over vinduet vises på computeren.

Man får 1 point for at overhale den gule bil, 2 point for at overhale den pink ATV og 3 point for at overhale kampvognen. Når man er Game Over, kan man indtaste sit navn og få sin score på highscore listen, når man trykker gem starter spillet igen, såfremt man ikke ønsker at skrive sig på listen kan man blot lukke highscore vinduet og et nyt spil startes uden tilføjelser til highscore listen.
Hastigheden for de andre biler øges med tiden, således at spillet bliver mere udfordrende over tid.

Ved hjælp af highscore listen, er det muligt at konkurrere i spillet mod hinanden.