\section{Kildetekst}

"Spil manual" : 
Efter udviklingen af en masse \texttt{VHDL} til hardware implementeringen og for vores vedkommende, dertil svarende \texttt{C-}kode, skal man i Xilinx miljøet compile sit projekt for at få lavet en bit fil som man skal uploade til \texttt{FPGA} boardet. Når dette er gjort, er \texttt{LC3} computeren kørt i stilling/gjort klar til at kunne køre selve spillet ("softwaren"). 
Efter endt kompilering, hvilket principielt er en form for fejlfinding (idet der tjekkes om alle signaler og komponenter er korrekt sat sammen), skal man compile \texttt{C-}koden og derefter er man klar til at køre/spille spillet. Som beskrevet tidligere i rapporten, bruger vi et rat som input til at kommunikere med spillet og styre bilen. 
For at få en kort gennemgang af hvordan man bruger spillet, tjek venligst "Bruger Manual" afsnittet.   