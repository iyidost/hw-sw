\section{Videreudvikling}
\subsection{Hardware}
Der er latches i projektet, hvilke kan give fejl og gøre det meget svært at rette disse, til videreudvikling af hardwaren ville dette derfor være noget, der ville blive prioriteret. Opbygningen af ROMs kunne forbedres, da disse er implementeret og fungerer, men ikke er lavet efter en entydigt plan. Det kunne være interessant at benytte lydkortet, og lave simpel lyd i spillet som f.eks. feedback når en forhindring rammes.

\subsection{Spil/Gameplay}
Der er opnået produkt der kører og fungere som et bilspil, hvilke var målet med projektet. Derfor er der stadig mulighed for udvidelser af spillet.
Der kunne være bedre grafik, i og med, de 24 bit farver ikke bliver udnyttet til fulde som projektet er nu.
Flere baner med forskellige grafik på vejen var også en oplagt mulighed for en udvidelse til spillet.
En større database med flere data gemt, ud over point, f.eks tid, statistik over hændelser i spillet, hvis det var samme spiller der spillede, dette kunne også inddrage kurset \texttt{02323 Sandsynlighedsregning og Statistik}.
Power ups\footnote{Med power ups forstås objekter der kan samles på banen som giver ekstra muligheder for spilleren i en kort periode af tid.} blev også diskuteret, da der allerede er implementeret 1 knap på rattet der ikke bliver benyttet til noget, dette kunne for eksempel være slow motion eller immunitet i en kort periodeaf tid.

Dette ville muligvis også kræve en mere detaljeret brugerflade, med både score, power ups, bane og liv på skærmen. En anden udvidelse kunne være at have 2 spillere, enten på samme computer eller over nettet.

Ved udvidelse med flere baner ville det være ønskeligt at lave en gem/indlæs funktion til spillet, så en spiller kan vende tilbage til samme sted i spillet på et senere tidspunkt. Ved opstart af spillet ville det være ønskeligt at have en intro der tæller ned til spillet starter, det samme når man kolliderer med et andet objekt, således at spilleren kan nå at være klar når spillet starter.

En mere forfinet kollision detektion ville også være en ønskelig tilføjelse, da alle objekter har en bestemt størrelse, både i højden og bredden, alt efter hvor mangle tiles de svarer til. Derfor støder spilleren nogle gane sammen med modstander objekter, uden der er en faktisk berøring visuelt. En sådan kollisions detektion kunne f.eks. gøres med cirkler eller med per-pixel kollisions detektion, omend dette også ville benytte flere CPU cycles ved hver frame opdatering.

Det kunne også være interessant at tilføje våben til spillet således at det er muligt at skyde forhindringer ned for at undgå at ramme dem.