\section{Videreudvikling}

Hardware Videreudvikling
Der er latches i projektet, hvilke kan give fejl og gøre det meget svært at rette disse, så til videreudvikling af hardwaren var dette noget, der ville prioriteret. Der kunne gøres noget ved opbygningen af ROMs, da disse er implementeret og fungerer, men ikke er lavet efter en entydigt plan.
Tilslutte lydkortet, og lave simpel lyd i spillet.

Spil Videreudvikling
Der er opnået produkt der kører og fungere som et bilspil, hvilke var målet med projektet. Derfor er der stadig mulighed for udvidelser af spillet.
 Der kunne være bedre grafik, i og med, de 24 bit farver ikke bliver udnyttet til fulde som projektet er nu. Flere baner med forskellige grafik på vejen var også en oplagt mulighed for en udvidelse til spillet.
 En større database med flere data gemt, ud over point, f.eks tid, statistik over hændelser i spillet, hvis det var samme spiller der spillede.
 Power ups blev også diskuteret, da der allerede er implementeret 1 knap på rattet der ikke bliver benyttet til noget. Hvilke ville kræve mere detaljeret brugerflade, med både score, power ups, bane og liv på skærmen.
 En anden udvidelse kunne være at have 2 spillere, enten på samme computer eller over nettet.
 Ved udvidelse med flere baner, lave en save/load funktion til spillet, så en spiller kan vende tilbage på et senere tidspunkt.
 Ved opstart af spillet have intro der tæller ned til spillet starter, det samme når man kolliderer med et andet objekt.
 En mere forfinet kollision detektion, da alle objekter har en bestem størrelse, både i højden og bredden, alt efter hvor mangle tiles de svarer til. Derfor støder spilleren nogle gane sammen med modstander objekter, uden der er en faktisk berøring visuelt.
 
 %lave dem som punktform
