\section{Design valg}

\subsection{Wrapper}
Til LC3 systemet valgtes det at bruge wrappers til de vhdl moduler der både har et input og et output, dette gjorte det nemmere at samle padding og tri state buffers. Ved at lave en wrapper kunne der sendes signaler videre ned i LC3 systemet fra bussen. I nogle af wrapperner i systemet ligger der logik, da det virkede mere overskueligt at dele logikken op, så ikke alt logik lå i hoved VHDL modulet. Dette var hensigtsmæssigt da et problem med VGA delen opstod. På grund af denne opbygningen var muligt at isolere VGA'en og syntetisere den alene, da hele systemet er forholdsvis tidskrævende at syntetisere.

\subsection{Padding}
Padding er en metode hvorved længden af en bitstreng ændres. Da det var nødvendigt, at ændre bus adresse længden i f.eks. UART'en fra en længde på 16 bit ned til 8 bit, da det er den længde UART forventer. Dette gøres ved, at når UART wrapperen modtager et signal sender den kun de 8 mindst betydende bit videre ned til UART'en at arbejde med, og når et signal modtages fra UART'en paddes der med 0'er foran, så der bliver sendt et signal tilbage til bussen der er konstrueret således '00000000' samt UART signalet. Derved ender vi med et signal på 16 bit igen, som er den længde bussen arbejder med. Det var muligt både at padde foran eller bagved et signal, alt efter om det var ønsket at bruge de mindst betdyende, mest betdyende eller nogle bestemte bit fra signalet.

%Eksempel fra koden

\subsection{Tri State Buffer}
En tri state buffer fungerer således at der bliver sendt et enable signal og et andet signal der angiver om der må skrives eller læses fra et modul, og derudover sørger for at hvis det ikke er det element der skal bruge signalet, kun vil modtage Z'er der i et signal er en tom værdi. På denne måde er det muligt at styre, hvilke elementer der bliver skrevet og læst fra på en given clock cycle. Dette kan give problemer hvis der bliver læst fra 2 moduler samtidig, og kan være svært at spore fejlen hvis ikke disse fungerer efter hensigten.

%Eksempel fra koden + diagram

\subsection{Chip Selector}
Chip selector gør det muligt at lave en kontrol af hvilke adresser der bliver skrevet til, og igennem et I/O adresse register%henvis til bilag
, er det blevet bestemt hvilke værdier der har med hvilke signaler at gøre. Ved at tjekke på denne adresse kan man derved bestemme om et signal til tri state bufferen skal være 1 eller 0.
%Billede af chip selector