\section{UART}
UART er en forkortelse for Universal asynchronous reciever and transmitter. Vi gik ud fra eksemplet i bogen, og valgte at implementere UART'en da det ville gøre vores debugging meget nemmere.  Vi valgte at lave en wrapper til UART'en da vores bus signal er 16 bit langt og det signal vi bruger i UART'en kun er 8 bit. Dette blev gjort ved at når der sendes til UART'en bruges kun (7 downto 0), altså de 8 mindst betydende bits. Og hvis i signal sendes fra UART'en paddes der med 0'er foran de 8 sendte bit. %Eventuelt bare lave et afsnit om vores wrapper/ padding da det går igen i en del af elementerne.
Vi brugte UART'en til at teste om Rammen og CPU kommunikerede korret sammen, via et test program givet til os, dette var et echo program der skrev de input tastaturet fik ud på skærmen.