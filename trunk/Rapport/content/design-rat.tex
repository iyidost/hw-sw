\section{Rat}
Det er valgt at benytte et rat som input, rattet er et gammelt playstation/pc rat som er skilt ad for at bruge rattets egenskaber og komponenter. Når rettet drejes påvirkes en variabel modstand som giver et analogt signal om rattets position, signalet læses af en 12 bit A/D converter\footnote{En analog til digital converter konventerer et analogt signal til et digitalt signal således at FPGA boardet kan læse fra ikke-digitale komponenter.} og kan på den måde fortolkes af LC3 processoren, når rattet er drejet helt til venstre er outputtet fra A/D converteren \textbf{0} mens at outputtet er \textbf{4096} når rattet er drejet helt til højre, A/D converteren er forbundet til et interface board fra Digilent\textregistered\footnote{Digilent FX2 MIB}

Der er desuden implementeret andre dele af rattets input og output funktionaliteter; en knap på rattet er forbundet til FPGA boardet igennem samme interface board, knappen er dog ikke i brug i det endelige projekt. Endvidere er to vibratorer fra rattet også styret af FPGA boardet således at det kan gives feedback når man kører ind i en forhindring, styringen af disse er udviklet i samarbejde med nogle diplom elektro studerende\footnote{Lasse Møller, s093440 og Emil Møller, s08310 hhv. 3. semester diplom elektro studerende og 5. semester diplom elektro studrende.}.