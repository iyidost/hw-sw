% nem indsættelse af billeder, eksempelvis \billede{!bthp}{0.9}{filnavn}{caption}
\newcommand*{\billede}[4]{
    \begin{figure}[#1]\centering
    \includegraphics[width=#2\textwidth]{billeder/#3}
    \caption{#4}\label{fig:#3}
    \end{figure}
}

\newcommand*{\billedea}[5]{
    \begin{figure}[#1]\centering
    \includegraphics[width=#2\textwidth]{billeder/#3}
    \caption[#5]{#4}\label{fig:#3}
    \end{figure}
}

\newcommand*{\plot}[5]{
    \begin{figure}[#1]\centering
    \includegraphics[width=#2\textwidth]{billeder/#3}\\
    \begin{tabular}{|r l|}
    	\hline
    	#5
    	\hline
    \end{tabular}
    \caption{#4}\label{fig:#3}
    \end{figure}
}

\newcommand*{\lline}[2]{
		\includegraphics{billeder/legend_#1} & #2 \\[0.1cm]
}

\newcommand*{\re}[1]{
		\textcolor{red}{\large{\textbf{#1}}}
		\normalsize
}

\newcommand*{\rea}[1]{
		\textcolor{blue}{\large{\textbf{#1}}}
		\normalsize
}

\newcommand*{\reb}[1]{
		\textcolor{green}{\large{\textbf{#1}}}
		\normalsize
}

\newcommand*{\rec}[1]{
		\textcolor{red}{\large{\textbf{#1}}}
		\normalsize
}

\newcommand*{\nn}{
		\nonumber
}

\newcommand*{\fkt}[1]{
		\textbf{#1}
}

\newcommand*{\ve}[1]{
		\vec{#1}
}

\newcommand*{\var}[1]{
%	\textsf{#1}
\texttt{#1}
}

\usepackage{array}
\newcolumntype{x}[1]{%
>{\raggedleft\hspace{0pt}}p{#1}}%

\newcommand{\tn}{\tabularnewline}

\newcommand*{\hs}{\hspace*{0.5cm}}%
