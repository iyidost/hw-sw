\chapter{Analyse og design}\label{cha:analysedesign}
Vores spil er udviklet på PC med C kode, derefter skal det overføres til FPGA board vha. en Serial forbindelse. Indholdet er gemmes i block ram på FPGA, og derefter vises på en VGA skærm, for man kan se og følge med hvad det sker og dermed kan spille. Spillet kan styres direkt fra FPGA board vha. ratten som vi har koblet til LC-3 system. 

LC-3 system efter vores design består af xx forskellige komponenter: CPU, rat, MEM, UART og  busser som forbinder dem sammen. Man kan kobler flere hardware enheder til hvis der er behøv for det. De kan være lydkort, VGA skærm o.s.v.

CPU har til formålet bruges til ....
MEM bruges til ...
UART bruges til ....
rat bruges til at styre billen som køre på forgrund.


Vi skal selv tegne LC-3 system diagram som ligner denne figur

\billede{!htbp}{1}{lc3system}{LC-3 System}