\chapter{Diskussion}\label{cha:diskussion}
Igennem 3 uger projektet er der blevet arbejdet med at designe og implementere et 2D bilspil. Der er blevet lagt en stor arbejdsindsats fra gruppen, blandt andet ved at arbejde i weekenderne, og derfor har det været muligt at nå minimumsimplementeringen tidligt i forløbet. Dette har gjort at der er blevet tilføjet flere features løbende. Spiller objektet er gået fra at være en rød kasse, der kunne bevæge sig til højre og venstre på skærmen til at være en tegnet sprite, der forestiller en racerbil med 2 forskellige 24 bit farver. Objekterne spilleren skal undgå, er ligesom spiller objektet, gået fra at være en boks til at være en sprite i 2 forskellige 24 bit farver. De modstandere der blevet tegnet er en guld nissan, en lilla ATV, og en army grøn tank.

Istedet for at styre spillerobjektet med knapper på DIO4 I/O boardet, er det muligt at styre spillerobjektet med et modificeret PS2/PC rat der er tilsluttet via en A/D converter, det har ikke var været så svært som forventet at implementere brug af rattet og har givet meget positiv gavn til spil oplevelsen.
 
Der er også blevet udviklet en highscore applikation, der kører på en lokal pc, der via serial kablet kan sende spillerens score til en database hvor der gemmes navn og score, dette trækker på viden fra kurset \texttt{02344 OOAD og Databaser}, denne applikation blev udviklet forholdsvist sent i udviklingsfasen og blev også udviklet på forholdsvis få arbejdstimer, men har udvist at være en positiv tilføjelse til projektet da det således at gjort nemmere at konkurrere mod hinanden.
 
I projektet var der problemer med at konvertere længden af bus signalet, dette blev løst ved at bruge padding.
 
Der var en del udfordringer under vejs, nogle at de mest tidskrævende var at få UART og RAM til at virke efter hensigten, da disse var nødvendige for at kunne uploade noget til FPGA boardet, dette blev dog gjort nemmere ved de udleverede testprogrammer, så fokus ikke var på udviklingen at brugbare testprogrammer, og at der dermed har været færre fejlkilder. Derfor var det også nødvendigt at bruge nogle design valg, der gjorte det muligt at sende og modtage fra disse to VHDL moduler til bussen. Da dette var på implementeret kunnet udviklingen af spillet og dets logik begynde. 

Da det grundlæggende i LC3 computeren var på plads og det var muligtat uploade kode over seriel forbindelsen begyndtes arbejdet med spillogik og anden styring fra C, den basiske logik blev færdig på forholdsvist kort tid derefter, hvorefter der blev brugt en masse tid på ekstra features, optimeringer samt fejl retninger.

% Grafik: Masker på en bedre måde? Skal evt. under videreudvikling
 
\chapter{Videreudvikling}\label{cha:videreudv}

