\chapter{Diskussion}\label{cha:diskussion}

Igennem 3 uger projektet er der blevet arbejdet med at designe og implementere et 2 D bilspil. Der er blevet lagt en stor arbejdsindsats fra gruppen,, blandt andet ved at arbejde i weekenderne, og derfor har det været muligt at nå minimumsimplementeringen tidligt i forløbet. Dette har gjort at der er blevet tilføjet flere features løbende. Spiller objektet er gået fra at være en rød kasse, der kunne bevæge sig til højre og venstre på skærmen til at være en tegnet sprite, der forestiller en racerbil med 2 forskellige 24 bit farver. Objekterne spilleren skal undgå, er ligesom spiller objektet, gået fra at være en boks til at være en sprite i 2 forskellige 24 bit farver. De modstandere der blevet tegner er en guld nissan, en lilla ATV, og en army grøn tank.
 I stedet for at styre spillerobjektet med knapper på DIO4 I/O boardet, er det muligt at styre spillerobjektet med et modificeret PS2/PC rat der er tilsluttet via en A/C converter. Dette var ikke så svært som forventet, og giver meget til spil oplevelsen at man kører en bil.
  Der er også blevet udviklet en highscore applikation, der kører på en lokal pc, der via serial kablet kan sende spillerens score til en database hvor der gemmes navn og score, dette trækker på viden fra kurset 02344 OOAD og Databaser.
  I projektet var der problemer med at konvertere længden af bus signalet, dette blev løst ved at bruge padding.

 Der var en del udfordringer under vejs, nogle at de mest tidskrævende var at få UART og RAM til at virke efter hensigten, da disse var nødvendige for at kunne uploade noget til FPGA boardet, dette blev dog gjort nemmere ved de udleverede testprogrammer, så fokus ikke var på udviklingen at brugbare testprogrammer. Derfor var det nødvendigt at bruge nogle design valg, der gjorte det muligt at sende og modtage fra disse to VHDL moduler til bussen. Da dette var på implementeret kunnet udviklingen af spillet og dets logik begynde. 
 Da det grundlæggende i LC3 computeren var på plads, begyndtes arbejdet med spillogik og anden styring fra C, dette gjorte arbejdet nemmere. % da vi har Mini the codemonkey!
 
\section{Videreudvikling}
\subsection{Hardware}
Der er latches i projektet, hvilke kan give fejl og gøre det meget svært at rette disse, til videreudvikling af hardwaren ville dette derfor være noget, der ville blive prioriteret. Opbygningen af ROM kunne forbedres, da disse er implementeret og fungerer, men ikke er lavet efter en entydigt plan. Det kunne være interessant at benytte lydkortet, og lave simpel lyd i spillet som f.eks. feedback når en forhindring rammes.

En betydelig forbedring på grafikken kunne være at lave en masking ROM og en farve ROM, og sørge for at der kunne sendes signaler dertil, indeholdende information om fra om hvilken ROM der skulle læses fra. Dette signal skulle kunne sendes fra C, og derved generere random modstandere med random farve. Derudover ville det give mulighed for objekterne kan indeholde flere farver. 2 bit ville give mulighed for 4 farver af 24 bit.

Klar deling af forskellige moduler, blandt andet tri state buffer der ligger i hoved VHDL modulet. Dette burde ligge under tilhørende wrapper.

\subsection{Spil/Gameplay}
Der er opnået produkt der kører og fungere som et bilspil, hvilke var målet med projektet. Derfor er der stadig mulighed for udvidelser af spillet.
Der kunne være bedre grafik, i og med, de 24 bit farver ikke bliver udnyttet til fulde som projektet er nu.
Flere baner med forskellige grafik på vejen var også en oplagt mulighed for en udvidelse til spillet.
En større database med flere data gemt, ud over point, f.eks tid, statistik over hændelser i spillet, hvis det var samme spiller der spillede, dette kunne også inddrage kurset \texttt{02323 Sandsynlighedsregning og Statistik}.
Power ups\footnote{Med power ups forstås objekter der kan samles på banen som giver ekstra muligheder for spilleren i en kort periode af tid.} blev også diskuteret, da der allerede er implementeret 1 knap på rattet der ikke bliver benyttet til noget, dette kunne for eksempel være slow motion eller immunitet i en kort periode af tid.

Dette ville muligvis også kræve en mere detaljeret brugerflade, med både score, power ups, bane og liv på skærmen. En anden udvidelse kunne være at have 2 spillere, enten på samme computer eller over nettet.

Ved udvidelse med flere baner ville det være ønskeligt at lave en gem/indlæs funktion til spillet, så en spiller kan vende tilbage til samme sted i spillet på et senere tidspunkt. Ved opstart af spillet ville det være ønskeligt at have en intro der tæller ned til spillet starter, det samme når man kolliderer med et andet objekt, således at spilleren kan nå at være klar når spillet starter.

En mere forfinet kollision detektion ville også være en ønskelig tilføjelse, da alle objekter har en bestemt størrelse, både i højden og bredden, alt efter hvor mangle tiles de svarer til. Derfor støder spilleren nogle gane sammen med modstander objekter, uden der er en faktisk berøring visuelt. En sådan kollisions detektion kunne f.eks. gøres med cirkler eller med per-pixel kollisions detektion, omend dette også ville benytte flere CPU cycles ved hver frame opdatering.

Det kunne også være interessant at tilføje våben til spillet således at det er muligt at skyde forhindringer ned for at undgå at ramme dem.

