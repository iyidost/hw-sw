\section{Ram}
 LC3 complete system RAM implementering.
Til projektet valgtes det at bruge single-port synchronous RAM. Der blev valgte single-port RAM fordi i første omgang var der ikke brug for at skrive til eller læse fra 2 adresser samtidig, som duel-port RAM ville have givet mulighed for. Synchronous tillader at bruge block RAM på FPGA boardet, og ikke CLBs RAM, der er forbeholdt til logic og asynchronous RAM. Til dette projekt var der ingen grund til at bruge asynchronous, da alle udregningerne ikke behøver at blive behandlet sammen clock cycle. Implementeringen af RAM, skete ved at følge bogen %(indsæt eksempel/reference) 
og derefter justere længden af adressen så det passede til den størrelse af det designede program der blev kørt på FPGA'en havde. Ved at reducere størrelsen på RAM, sørgede dette for at syntetiseringen af LC3 systemet ikke tog unødvendig langt tid. Der blev dog løbende ændret på adressen længden af hukommelsen, da C kode blev længere, og krævede en større hukommelse. %Hvad bruger vi rammen til ud over vores c kode? 

Til design af RAM blev der brugt følgende design decisions. Der blev lagt en wrapper omkring RAM for at gøre det nemmere at bruge et andet design decicions nemlig padding. Derudover er benyttet en tri state buffer til at afgøre om der skulle læses eller skrives til rammen.
% henvis til design decicions i starten af design afsnittet.