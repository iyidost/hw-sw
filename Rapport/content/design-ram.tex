\section{Ram}
LC3 complete system RAM implementering.
Til projektet valgtes det at bruge single-port synchronous RAM. Der blev valgt single-port RAM på baggrund af at der i første omgang ikke var brug for at skrive til eller læse fra 2 adresser samtidig, som dual-port RAM ville have givet mulighed for. Synchronous tillader at bruge block RAM\footnote{En block ram er en dedikeret to-ports ram på FPGA'en} på FPGA boardet, og ikke CLB RAM\footnote{Configurable Logic Block}, der er forbeholdt til logik og asynchronous RAM. Til dette projekt var der ingen grund til at bruge asynchronous, da der ikke er behov fra paralelle udregninger i samme clock cycle i dette projekt. Implementeringen af RAM skete ved at følge bogen %(indsæt eksempel/reference) 
og derefter justere længden af adressen så det passede til den størrelse af det designede program der blev kørt på FPGA'en. Ved at reducere størrelsen på RAM sørgede dette for at syntetiseringen af LC3 systemet ikke tog unødvendigt langt tid. Der blev dog løbende ændret på adresse længden af hukommelsen, da C koden blev længere, og krævede en større hukommelse. Det er muligt at ligge C kode op på RAM, så .exe-filen bliver unødvendig, og der kun er brug for .bit-filen for at køre projektet.

Til design af RAM blev der brugt følgende design valg: Der blev lagt en wrapper omkring RAM for at gøre det nemmere at benytte padding. Derudover er der benyttet en tri state buffer til at afgøre om der skulle læses eller skrives til RAM, som beskrevet i afsnittet design valg.
% henvis til design decicions i starten af design afsnittet.