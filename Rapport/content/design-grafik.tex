\section{Grafik}
Til projektet efter minimum implementeringen var udført, blev grafikken til spillet udvidet. 

\subsection{Sprites}
 Dette skete ved at der blevet tegnet nogle png filer i photoshop, der derefter blev konverteret til et hex array ved hjælp af værktøjeret\footnote{matisen.dk/dtu/02321}, udviklet i forbindelse med projektet. Disse Hex arrays blev lagt ind i en VGA\_ROM som tegnede rammen for objektet, derefter blev der ud fra png filen lave en inverted udgave som blev gemt i en anden ROM som udgjorte masken til objektet. Til hver af disse ROM der tildelt 1 farve hvor de værdier der er sat til 1 fik denne farve. Ved hjælp af denne metode kunne rammen og masken få 2 forskellige farver og danne et objekt med 8 bit farver.
 
\subsection{Tiles}
Tiles blev lavet ved hjælp af værktøjet\footnote{http://matisen.dk/dtu/02321/tile\_maker.php}, udarbejdet i forbindelse med projektet. Disse hex array blev lagt ind i en Tile ROM, hvor værdien 0 eller 1 bestemmer farven på tilen ud fra en anden ROM hvor farven til en bestemt tile er defineret.
