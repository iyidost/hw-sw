\chapter{Opgaveformulering}\label{cha:opgaveformulering}
\section{Case}
Vi skal til vores 3 ugers projekt i HW/SW udvikle et kørende program på vores FPGA board, hvor vi har valgt at udvikle et køre-bil simulator spil. Vi ønsker at have noget simpel grafik der bevæger sig, altså en bil der kører "ud af vejen", der skal undgå div. forhindringer. Vi ønsker at tilslutte et rat som styrings element til vores bil via com. port og hvis dette ikke kan lade sig gøre, vil vi tilslutte det til computeren og kommunikere til boardet via seriel port.
\billede{!htbp}{0.7}{MockUpSimple}{Simpelt mock-up over systemet}
(den røde kasse symboliserer bilen som kan styre hhv. til højre og venstre, ved drej på det tilsluttede rat. De blå kasser symboliserer forhindringer på vejen, rammer man dem er spillet slut. Forhindringerne vises tilfældigt.)
Neden for ses et mere avanceret mock-up over systemet, hvor vi, afhængigt af tiden op til aflevering, påtænker at indsætte flere forhindringer på vejen, påtænker flere forskellige baner at spille, skiftende baggrunde og bedre grafik. Vi begynder i det små, får de forskellige komponenter til at fungere korrekt sammen og udvikler derefter spillet til FPGA'en. Vi stiler efter, at nå mest muligt af vores ønskede krav iht. vores tænkte avancerede udgave af spillet og forbedre vores forståelse for sammenspillet mellem hardware og software udvikling. Der ønskes også at vi bruger lidt teknologier fra nogle af de andre fag, f.eks. noget DB til at gemme highscores i (med henvisning til OOAD faget) eller lidt statistik over highscores(med henvisning til Statistik faget) osv.
\billede{!htbp}{0.7}{MockUpAdvanced}{Avanceret mock-up over systemet}